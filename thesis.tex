\documentclass[11pt,english]{toptesi}

%%%%%%%%%%%%%%%%%%%%%%%%%%%%%%%%%%%%%%%%%%%%%%%%%%%%%%%%%%%%%%%

\usepackage[utf8]{inputenc} %utf8
\usepackage[english]{babel}
\usepackage[T1]{fontenc}
\usepackage{blindtext}
\usepackage{graphicx,wrapfig}
\usepackage{booktabs}
\usepackage{lmodern}
\usepackage{varioref}
\usepackage{url}
\usepackage{array}
\usepackage{paralist}{\obeyspaces\global\let =\space}
\usepackage{verbatim} 
\usepackage{subfig}
\usepackage{tabularx}
\usepackage{amsmath}
\usepackage{amsfonts}
\usepackage{float}
\usepackage{amssymb}
\usepackage{multicol}
\usepackage{multirow}
\usepackage{listings}
\usepackage[pass]{geometry}
\usepackage[figuresright]{rotating}
\usepackage{algorithm}
\usepackage{algorithmic}
\usepackage{amsmath}
\usepackage[babel]{csquotes}
\usepackage{hyperref}
\usepackage[backend=bibtex]{biblatex}

%%%%%%%%%%%%%%%%%%%%%%%%%%%%%%%%%%%%%%%%%%%%%%%%%%%%%%%%%%%%%%%

% CONFIGURAZIONE LINK E RIFERIMENTI
\hypersetup{%
    pdfpagemode={UseOutlines},
    bookmarksopen,
    pdfstartview={FitH},
    colorlinks,
    linkcolor={black}, %COLORE DEI RIFERIMENTI AL TESTO
    citecolor={blue}, %COLORE DEI RIFERIMENTI ALLE CITAZIONI
    urlcolor={blue} %COLORI DEGLI URL
}

%%%%%%%%%%%%%%%%%%%%%%%%%%%%%%%%%%%%%%%%%%%%%%%%%%%%%%%%%%%%%%%

% CONFIGURAZIONE LISTATI/CODICE - CANCELLARE SE NON NECESSARIO
% PYTHON - BIANCO E NERO
\lstset{%
	captionpos=b,
	language=Python,
	basicstyle =\small\ttfamily,
	keywordstyle=\color{black}\bfseries,
	breaklines=true,
	breakatwhitespace=true,
	frame=lines,
	numbers=left,
	numberstyle=\footnotesize,
}

%%%%%%%%%%%%%%%%%%%%%%%%%%%%%%%%%%%%%%%%%%%%%%%%%%%%%%%%%%%%%%%

%DEFINIZIONE SEZIONI IN NUMERAZIONE ROMANA
%ELENCO DEI LISTATI/CODICI
\makeatletter
\newcommand\listofcodes{%
 \iffrontmatter\else\frontmattertrue\fi
 \if@openright\cleardoublepage\else\clearpage\fi
 % change the meaning of \chapter in a group
 \begingroup\def\chapter##1{\@schapter}
 \phantomsection % for the hyperlink
 \lstlistoflistings 
 \endgroup
} 
\makeatother

%%%%%%%%%%%%%%%%%%%%%%%%%%%%%%%%%%%%%%%%%%%%%%%%%%%%%%%%%%%%%%%

% INFORMAZIONI PDF - PERSONALIZZARE
\pdfinfo{%
  /Title    (MPTCP Security Evaluation)
  /Author   (Fabrizio Demaria)
  /Subject  (Analysing and fixing critical MPTCP vulnerabilities, contributing to the Linux kernel implementation of the protocol)
  /Keywords (MPTCP Security)
}

%%%%%%%%%%%%%%%%%%%%%%%%%%%%%%%%%%%%%%%%%%%%%%%%%%%%%%%%%%%%%%%

% FRONTESPIZIO - PERSONALIZZARE
% ELIMINATE LE VOCI CHE NON VI SERVONO

% UNIVERSITA - NOME
\ateneo{Politecnico di Torino / Stockholm's KTH}

% FACOLTA - DICITURA
\FacoltaDi{Faculty of }
% FACOLTA - NOME
\facolta{Engineering}

% CORSO DI LAUREA - DICITURA (MANTENERE LO SPAZIO)
\CorsoDiLaureaIn{Master of Science in }
% CORSO DI LAUREA - NOME
\corsodilaurea{Computer Science}

% TIPOLOGIA TESI
\TesiDiLaurea{Master Thesis}

% TITOLO
\titolo{MPTCP Security Evaluation}

% SOTTOTITOLO
\sottotitolo{Analysing and fixing critical MPTCP vulnerabilities, contributing to the Linux kernel implementation of the protocol}

% RELATORE/I - DICITURA
\AdvisorName{Advisors}
% RELATORE - PROF. NOME E COGNOME
\relatore{prof.\ Antonio Lioy}
% RELATORE AGGIUNTIVO - PROF NOME E COGNOME
% SE SI HA SOLO UN RELATORE ELIMINARE E CAMBIARE Advisors in Advisor
\secondorelatore{prof.\ Peter Sjödin}

% TUTORE AZIENDALE - TITOLO NOME E COGNOME
\tutoreaziendale{Henrik Svensson\\Joakim Nordell\\Shujuan Chen}
% TUTORE AZIENDALE - DICITURA//AZIENDA
\NomeTutoreAziendale{Company tutors\\Intel Corporation Inc}

% CANDIDATO - DICITURA (MANTENERE I DUE PUNTI)
\CandidateName{Candidate:}
% SECONDO CANDIDATO - ELIMINARE O DECOMMENTARE
%secondocandidato{Bombo de Bombis}

% CANDIDATO - NOME E COGNOME
\candidato{Fabrizio Demaria}

% LOGO UNIVERSITA
\logosede{images/logo}

% DATA - MESE ANNO
\sedutadilaurea{March 2016}

%%%%%%%%%%%%%%%%%%%%%%%%%%%%%%%%%%%%%%%%%%%%%%%%%%%%%%%%%%%%%%%

% LISTA DEI CAPITOLI DA INCLUDERE - PERSONALIZZARE
\includeonly{%
chap_introduction,%
chap_multipathtcp,%
chap_mptcpsecurity,%
chap_addaddrattackexecution,%
chap_addaddr2,%
chap_conclusions,%
app_a,%
}

% FILE DI BIBLIOGRAFIA
\bibliography{bibliography} 


%%%%%%%%%%%%%%%%%%%%%%%%%%%%%%%%%%%%%%%%%%%%%%%%%%%%%%%%%%%%%%%

% INIZIO DOCUMENTO
\begin{document}
\english

\frontespizio

%%%%%%%%%%%%%%%%%%%%%%%%%%%%%%%%%%%%%%%%%%%%%%%%%%%%%%%%%%%%%%%

%INTERLINEA - DEFAULT 1
%\interlinea{1.2}

%%%%%%%%%%%%%%%%%%%%%%%%%%%%%%%%%%%%%%%%%%%%%%%%%%%%%%%%%%%%%%%

\frontmatter

% DEDICA - PERSONALIZZARE
% VSPACE - PROPORZIONE USATA PER CENTRATURA VERTICALE DEL TESTO
% FLUSHRIGHT - ALLINEAMENTO ORIZZONTALE A DESTRA
%\vspace*{\stretch{1}}
%\begin{flushright}
%\noindent
%To...
%\end{flushright}
%\vspace*{\stretch{6}}
%\cleardoublepage

% CITAZIONE - PERSONALIZZARE
% VSPACE - PROPORZIONE USATA PER CENTRATURA VERTICALE DEL TESTO
% FLUSHRIGHT - ALLINEAMENTO ORIZZONTALE A DESTRA
%\vspace*{\stretch{1}}
%\begin{flushright}
%\noindent
%Citation

%\textit{cit.}
%\end{flushright}
%\vspace*{\stretch{6}}
%\cleardoublepage

%%%%%%%%%%%%%%%%%%%%%%%%%%%%%%%%%%%%%%%%%%%%%%%%%%%%%%%%%%%%%%%

% RINGRAZIAMENTI - PERSONALIZZARE
\ringraziamenti
Thanks to...

%%%%%%%%%%%%%%%%%%%%%%%%%%%%%%%%%%%%%%%%%%%%%%%%%%%%%%%%%%%%%%%

% ABSTRACT - PERSONALIZZARE
\sommario
Abstract goes here...

%%%%%%%%%%%%%%%%%%%%%%%%%%%%%%%%%%%%%%%%%%%%%%%%%%%%%%%%%%%%%%%

% INDICI - ELIMINARE GLI INDICI NON NECESSARI

% INDICE GENERALE
\tableofcontents

% INDICE DELLE FIGURE
%\listoffigures

% INDICE DELLE TABELLE
%\listoftables

% INDICE DEI CODICI
%\listofcodes

%%%%%%%%%%%%%%%%%%%%%%%%%%%%%%%%%%%%%%%%%%%%%%%%%%%%%%%%%%%%%%%

\mainmatter

% INCLUSIONE FILE CAPITOLI - PERSONALIZZARE - TENERE COERENTE CON LISTA IN ALTO
\chapter{Introduction}
\label{chap:introduction}

The introductory part is going to cover MPTCP from a general perspective, taking into consideration all the possible ways to take advantage of the new protocol and all the improvements (benefits) observable by the end users with respect to the old, common TCP. Problem statement and thesis' objectives follow.

\section{Motivation}
This section would start with a general introduction of the interconnected world of today, discussing how hardware and software communication has changed in the last decade. The focus of this part is to bring up the multihoming and multipath reality of the infrastructures of today and how this led almost naturally to the MultipathTCP project. It would be good to cite similar technologies developed before MPTCP (for example SCTP), explaining in which aspects of MPTCP is supposed to be a better option.
This should include an overview of the real benefits that can be achieved by adopting MPTCP in common appliances (smartphones for example) as well as modern datacenters. It would be good to explain the fact that MPTCP was designed to be as retrocompatible as possible with current infrastructure (lower layer) and applications (higher layer); this is a good way to introduce the next section, where compatibility issues are presented.

\section{Problem Statement}
After a general introduction of the protocol, here it follows the problem statement related to this thesis .
This part should show the context of the overall content, namely the security analysis of MPTCP and the specific case of the ADD\_ADDR attack, considered a blocking issue in the upstreaming process of the MPTCP implementation.
This section should contain the objectives of the thesis: primarily fixing the ADD\_ADDR vulnerability of the protocol, developing effective and powerful simulation scenarios in order to easily test MPTCP by using UML and Scapy and finally contributing to the upstreaming of the protocol into the Linux kernel by developing patches and improving official RFC documentation.

\section{Methodology}
This section should contain a short road map containing the various step taken to fix the problems and the general methodology adopted, using a top-down approach.
Perhaps, it is possible to cite here the working environment and the parties involved. This section might also contain an explanation of the structure of the text.

\chapter{Multipath TCP}
\label{chap:multipathtcp}

From now on the discussion becomes more technical. This chapter is all about how MPTCP works (without referring to a specific implementation).

\section{Transmission Control Protocol (TCP)}
Introduction of TCP and how this old, established protocol works. This is a good starting point from where the MPTCP extension discussion can start (in the next section).

\section{Extending TCP to MultiPath TCP}
How MPTCP is added on top of TCP (with all the related design aspects) is reported here. This is the first portion of the thesis containing a more in-depth description of the protocol. This part might follow closely the introductory portions of the RFC documents regarding MPTCP.

\subsection{Control Plane}
All the MPTCP options used to manage MPTCP sessions are reported and explained here, including all the details on how to set a new session and add/remove subflows.

\subsection{Data Plane}
This part concerns all the MPTCP options used to manage the data flow in a MPTCP session, including how the byte stream is subdivided into different subflow and how the original order of the packets is provided at the receiver.

\section{MPTCP Deployment}
\subsection{Middleboxes Compatibility}
This section will be quite technical and it is supposed to list the most important middle-boxes and their impact/effect on a MPTCP connection. These boxes include NATs, proxies and firewalls. This part should clearly state why MPTCP widespread adoption is a big challenge.

\subsection{Implementations}
Despite the previously described problematics, MPTCP is a big bet in the IETF community and many implementations have been developed for the most common OSes, listed in this section (with some history notions).

\subsection{Deployment Status}
It should be interesting for the reader to go through some examples of real world's scenarios in which MPTCP is used successfully. Here it is possible to cite some important achievements related to MPTCP (for example the highest throughput ever reached with the new protocol).
\chapter{MPTCP Security}
\label{chap:mptcpsecurity}

This chapter starts with a general overview and it later introduces the theory behind the residual threats that affect MPTCP, according to the most recent documents and research.

\section{Threat Analysis}
A general introduction about the security requirements elaborated for MPTCP is reported here. This part is also supposed to present some categorizations related to general networking attacks, in order to give a good idea of the possible threats and their effects on an ongoing connection (not only for the MPTCP case). These notions are later mentioned again when listing the various attacks to which MPTCP is currently vulnerable.

\section{ADD\_ADDR Attack} \label{theaddaddrattack}
The most important attack is the ADD\_ADDR attack. It is the most dangerous and in the end it is the main topic of the whole work carried out for this thesis. This section explains in details the theory behind the attack as well as the steps to be followed in order to carry out the attack. No simulation is cited here, since an entire chapter is dedicated to that later on.

\subsection{Concept}
\subsection{Procedure}
\subsection{Requirements}


\section{Additional Threats}
Even if the paper focuses on ADD\_ADDR attack, it is a good point to present here the other residual threats that are reported in RFC 7430. These are considered minor threats.

\subsection{DoS Attack on MP\_JOIN}
\subsection{Keys Eavesdrop}
\subsection{SYN/ACK Attack}
\chapter{ADD\_ADDR Attack Execution}
\label{chap:addaddrattackexecution}

Since the first part of the thesis work has been devoted to build the attacking tool to reproduce the exploitation of the ADD\_ADDR vulnerability, an entire chapter is dedicated to this topic.

\section{Environment Setup}
Here it will be explained what UML virtual machines are and why they were good candidates for the simulation tests. The setup procedure is also reported here, with graphs to visually show the simulation's network scenario. The Scapy tool is also presented here. This was a great program to manipulate and forge packets.

\section{Attack Script}
This part is supposed to explain how the Scapy attacking script actually replicates all the steps needed to execute the ADD\_ADDR attack. The steps were explained from a theoretical point of view in the previous chapter. Some code snippets of the attacking script can be present in this section.
Some details on the problems encountered in building the script and how they have been solved can be valuable material, too.

\section{Reproducing the Attack}
This section is a step-by-step tutorial on how to call the attack script and reproduce the attack. It mainly contains the README file in the GitHub repository where the script can be retrieved.
Screenshots and/or Whireshark output can be added to show the actual behaviour of the attack over a netcat connection.

\section{Conclusions} 
A final evaluation of the experiment is needed. First, it is important to emphasize the limitations of the adopted tool and how this simulation differs with respect to real scenarios. Nevertheless, it is crucial to explain the value of such experiment, that indeed proves the feasibility of the ADD\_ADDR attack and better justify the development of a fix, which is the main topic of the next chapter.


\chapter{ADD\_ADDR2}
\label{chap:addaddr2}

This chapter is related to the actual work and original contribution performed during the Master Thesis work at Intel. The ADD\_ADDR2 option is developed and added to the current MPTCP implementation for the Linux kernel in order to fix the vulnerability.

\section{The ADD\_ADDR2 format}
The actual new format and the reasons why it fixes the vulnerability of ADD\_ADDR are reported here. Various discussions can be added to explain why this is believed to be the best way to fix the problem. This part doesn't include any implementation details and/or code snippets.

\section{Implementing ADD\_ADDR2}
An introductory section that shows the main architectural aspects of how MPTCP has been merged into the TCP code and the TCP modules inside the kernel.
Here it starts the part with all the details and work related to the implementation of ADD\_ADDR2 in the kernel. Code snippets have to be added here. The following subsections are the side issues and side features that have been elaborated during the thesis work.

\subsection{Retro-compatibility}
Version control mechanism was not present but it is needed to negotiate which format to use in a MPTCP session: ADD\_ADDR or ADD\_ADDR2.

\subsection{Port Advertisement}
Port advertisement in ADD\_ADDR is possible according to RFC specifications but it was not part of the implementation at the beginning of the thesis work, so it has been added.

\subsection{IPv6 Considerations}
Longer addresses brought some issues related to TCP option fields limitations.

\subsection{Crypto-API in MPTCP}
A major problem was how to deal with the new hashing requirements introduced by ADD\_ADDR2. Extending the current MPTCP hashing function to deal with input messages of arbitrary size is a first point to explain. The second part has to deal with the whole analysis related to adopting the kernel CRYPTO APIs to calculate the HMAC values in MPTCP and why this is not advisable.

\section{RFC Contributions}
Another minor part of the thesis work on MPTCP is related to some small contributions to the official RFC documentation.

\section{Experimental Evaluation}
This part should include performance analysis regarding the new format introduced with ADD\_ADDR2. A discussion on how the new format (and all the other modifications introduced with the patches) could impact any aspect of the protocol should be present in this section. At the moment, no data has been produced that might be a good candidate for this section.
\chapter{Conclusions}
\label{chap:conclusions}

\section{Related Work}
References to all the related work, including all the efforts to make MPTCP secure and stable.

\section{Future Work}
Listing the next steps to be taken care of in terms of MPTCP security, in order to facilitate the protocol upstream and widespread deployment.

\section{Final Thoughts}
Some short conclusion on the presented work.

\appendix
% INCLUSIONE APPENDICI - - PERSONALIZZARE - TENERE COERENTE CON LISTA IN ALTO
\chapter{An appendix}
\label{app:a}
Appendix content goes here...


%%%%%%%%%%%%%%%%%%%%%%%%%%%%%%%%%%%%%%%%%%%%%%%%%%%%%%%%%%%%%%%

% BIBLIOGRAFIA
\addcontentsline{toc}{chapter}{\refname}
\nocite{*}
\printbibliography

\end{document}