\section{User Mode Linux}
In order to achieve a reliable reproduction of a real world scenario, the simulation includes the setup of two User Mode Linux (UML) virtual machines running a Linux kernel with enabled support for MPTCP. These two machines acts as client and server, carrying on a MPTCP connection that is the target for the ADD\_ADDR attack. 
Using UML to proceed with the experiments allows for very fast setup and boot-up time, with good emulation of real machines and giving the possibility to work on a single hosting machine with no risk of damaging or crashing its underlying kernel.

A good resource in terms of tools, configuration files and kernel images is the official mptcp website:
\textit{http://www.multipath-tcp.org}. In particular, the website offers a python script that downloads all the necessary files to run the two virtual machines. Considering our purpose of verifying the ADD\_ADDR attack feasibility, we don't need to modify or debug the Linux kernel source code, and we can just use the above mentioned components out of the box. At this stage of our analysis it is actually advised to perform the attack on the official distribution as it is, and develop external tools for injecting packets and monitoring the status of the connections. More specifically, the MPTCP version adopted for our tests is: \textit{Stable release v0.89.0-rc}.

When executing the script \textit{setup.py} retrieved from the official Website, a few files are downloaded. A \textit{vmlinux} executable file with the MPTCP compatible Linux kernel, two file-systems for the client and the server (\textit{fs\_client} and \textit{fs\_server}) and two shell scripts to configure and run the virtual machines (\textit{client.sh} and \textit{server.sh}). No configuration is needed, and client and server should be able to connect via MPTCP right away.
Here it follows the content of the \textit{client.sh} (a similar shell script not reported here can be found for the server counterpart, including a single \textit{tap2} interface setup in that case):


\begin{lstlisting}[language=bash, caption=\textit{client.sh}]
#!/bin/bash

USER=`whoami`

sudo tunctl -u $USER -t tap0
sudo tunctl -u $USER -t tap1

sudo ifconfig tap0 10.1.1.1 netmask 255.255.255.0 up
sudo ifconfig tap1 10.1.2.1 netmask 255.255.255.0 up

sudo sysctl net.ipv4.ip_forward=1
sudo iptables -t nat -A POSTROUTING -s 10.0.0.0/8 ! -d 10.0.0.0/8 -j MASQUERADE

sudo chmod 666 /dev/net/tun

./vmlinux ubda=fs_client mem=256M umid=umlA eth0=tuntap,tap0 eth1=tuntap,tap1

sudo tunctl -d tap0
sudo tunctl -d tap1

sudo iptables -t nat -D POSTROUTING -s 10.0.0.0/8 ! -d 10.0.0.0/8 -j MASQUERADE
\end{lstlisting}

With these scripts, it is possible to run the two UML machines and use the local \textit{tap} interfaces to sniff and inject packets (acting, in this specific case, as a physical man in the middle).

The resulting network scenario is graphically depicted in Figure \ref{fig:networkscenario}.

\begin{figure}[!htb]
\centering
\includegraphics[width=\textwidth]{Network_Scenario}
\caption{Network scenario}
\label{fig:networkscenario}
\end{figure}

By establishing a testing connection via the \textit{iperf} tool, two subflows are automatically generated by MPTCP, from the two interfaces of client (ip addresses: \textit{10.1.1.2} and \textit{10.1.2.2}) and the single server's interface (with ip address: \textit{10.2.1.2}).

\section{Scapy tool}


\section{Results analysis}
\section{Limitations and future work}

